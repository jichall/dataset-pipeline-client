% Created 2019-03-01 sex 23:38
% Intended LaTeX compiler: pdflatex
\documentclass[a4paper, 12pt]{article}
\usepackage[utf8]{inputenc}
\usepackage[T1]{fontenc}
\usepackage{graphicx}
\usepackage{grffile}
\usepackage{longtable}
\usepackage{wrapfig}
\usepackage{rotating}
\usepackage[normalem]{ulem}
\usepackage{amsmath}
\usepackage{textcomp}
\usepackage{amssymb}
\usepackage{capt-of}
\usepackage{hyperref}
\usepackage{minted}
\usepackage{fancyhdr}
\usepackage{lipsum}
\usepackage{indentfirst}
\usepackage[portuguese, ]{babel}
\usepackage{libertine}
\usepackage{tkz-graph}
\usepackage[usenames,dvipsnames]{xcolor}
\usepackage[left=3cm,bottom=3cm,top=2cm,right=2cm]{geometry}
\newcommand{\code}{\texttt}
\pagestyle{empty}
\fancyfoot[R]{\thepage}
\author{Rafael Campos Nunes}
\date{\textit{<2019-03-01 sex>}}
\title{Dataset Pipeline Client}
\hypersetup{
 pdfauthor={Rafael Campos Nunes},
 pdftitle={Dataset Pipeline Client},
 pdfkeywords={},
 pdfsubject={},
 pdfcreator={Emacs 26.1 (Org mode 9.1.9)},
 pdflang={Pt-Br}}
\begin{document}

\maketitle
\tableofcontents

A elaboração de uma API que disponibiliza um serviço de upload de arquivos sobre
o protocolo HTTP se completa com a criação de um cliente onde as seguintes
funções foram determinadas:

\begin{itemize}
\item Leitura de um diretório de arquivos contendo arquivos JSON
\item Envio dos dados ao serviço onde serão persistidos de alguma maneira
\item Consulta dos dados através do atributo PK ao serviço disponibilizado
\end{itemize}

Ao realizar  as tarefas denotadas anteriormente a linguagem Go foi escolhida
para o desenvolvimento  visto que sua biblioteca padrão fornece vários recursos
que permitem a escrita da aplicação de maneira simples e fácil além de ser uma
linguagem de compilação rápida, contrastando essas características com
linguagens como C e C++. A serialização de extração de dados dos arquivos no
formato mencionado anteriormente é realizada utilizando o pacote \(encoding/json\),
 o envio destes dados pode ser feita utilizando o pacote \(net/http\) e a leitura
de diretórios/arquivos é realizada utilizando os pacotes \(os\), \(io\) e
\(io/ioutil\).

Ao ler o diretório bastou usar a função contida no pacote \(io/ioutil\) denominada
\(LoadDir\) onde é passada uma cadeia de caracteres representando o caminho do
diretório no fs relativo ao ambiente de execução do programa. A partir
disso é retornado um vetor de `os.FileInfo` onde é possível verificar algumas
características do arquivo como se é um diretório e o nome deste.

Ao filtrar os arquivos a serem lidos por extensão e também pela característica
de arquivo é denotada uma estrutura que armazena as informações do arquivo lido

Os arquivos são simplesmente arquivos de texto (a extensão não foi determinada)
que armazenam, a cada linha deste, um JSON (Javascript Object Notation) de tal
forma que a estrutura desse arquivo é tal como abaixo:

\begin{minted}[frame=lines,linenos=true]{javascript}
{ "pk": "1", "score": "321" }
{ "pk": "2", "score": "435" }
...
{ "pk": "101", "score": "13" }
\end{minted}

Observa-se que o objeto contém dois atributos, o atributo \(pk\) e o \(score\). É
fácil, portanto, definir uma estrutura em \uline{Go} (linguagem utilizada para
desenvolver a aplicação) para serialização, e vice versa com serialização
reversa também conhecida neste contexto como \uline{unmarshalling}\footnote{Serialization in Computer Science available on: \url{https://en.wikipedia.org/wiki/Serialization}}, de dados.

A extração da informação contida nos arquivos é feita utiliza ndo

O envio dos dados é realizado utilizando a API REST e o console. Ao utilizar a
função \emph{send} é executado o carregamento de arquivos dentro do diretório
definido previamente na váriavel \emph{dir} e, após isso, envia-se requisições GET
ao ponto da API com as informações necessárias.
\end{document}
